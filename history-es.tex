\textbf{Datos Personales}\\[6pt]
DU: 29006449.\\
Fecha de Nacimiento: 16 de agosto de 1981.\\
Domicilio: Vidal 2027 12 E, (1428) Buenos Aires. Argentina.\\
Celular: +54 9 11 5 040 0483.\\
E-mails: ajawtu@dm.uba.ar, alexis.jawtuschenko@gmail.com\\
Sitios Web: \href{https://github.com/alexisjawtu}{{\color{blue}Portfolio de GitHub}}
-- \href{https://www.linkedin.com/in/alexis-jawtuschenko/}{{\color{blue}LinkedIn}}

\textbf{T\'{i}tulos Universitarios}\\[6pt]
Doctorado de la Universidad de Buenos Aires, `area Ciencias Matem`aticas (2018).\\
Tema de tesis: Mallas Poliedrales Tridimensionales y An`alisis Num`erico Vectorial.\\
\href{https://cms.dm.uba.ar/academico/carreras/doctorado/thesisJawtuschenko.pdf}
{{\color{blue}Link a la tesis.}}

Licenciatura en Ciencias Matem`aticas de la Univesidad de Buenos Aires (2010).\\
Tema de tesis: Programaci`on Lineal Entera e Investigaci`on Operativa aplicada a
Subastas en Combinaciones.\\
\href{https://cms.dm.uba.ar/academico/carreras/licenciatura/tesis/2010/Jawtuschenko_Alexis.pdf}
{{\color{blue}Link a la tesis.}}

\textbf{Lenguajes Naturales}\\[6pt]
Fluidez en ingl`es y alem`an.

\textbf{Lenguajes de Programaci`on y Computaci`on}\\[6pt]
\Cpp,\;\;{Python/NumPy/Pandas}.

{GNU Octave/MATLAB},\;\;{IBM ILOG CPLEX},\;\;{Zimpl}.

{LaTeX}.

{GNU/Linux Debian},\;\;{Git},\;\;{Bash}.

\emph{Curso de Especializaci`on}\\
{Workshop en T\'ecnicas de Programaci\'on Cient\'ifica 2017} en FACET, 
Universidad Nacional de Tucum\'an, consistente en teor\'ia y aplicaci\'on de t\'ecnicas 
de desarrollo en C, Python, Fortran y C\masmas\,\,para problemas cient\'ificos.

Link a mi perfil \href{https://projecteuler.net/profile/alexisj.png}
en {\color{blue}{Project Euler}}.

\textbf{Antecedentes de trabajo en Programaci`on y Computaci`on}\\[6pt]
\Cpp. Desarrollador de Software de mallado y geometr`ia 3D en Niigata University (Japan).\\
De agosto 2022 a marzo 2023 como aut`onomo remoto.\\
De junio 2023 a agosto 2023 como Profesor-Investigador visitante Full Time.\\

Python/Pandas y CPLEX. Desarrollador de Software, especialista en
Optimizaci`on Combinatoria en Mercado Libre (Argentina). 
Full-Time de agosto de 2020 a junio de 2022.

Php 5 y PostgreSQL.  Desarrollador de Software en el
Ministerio de Educaci`on Argentina. Desde junio 2011 hasta
marzo 2012. 

ActionScript 3.0 y Python. Trainee en Programaci`on de Video Juegos en
DeveGo SA. (Argentina). Desde marzo 2011 hasta mayo 2011.

PHP, MySQL y HTML. Desarrollador Web Part Time en ViaRural.com SA. Desde junio 2008
hasta febrero 2011.

PHP, MySQL y HTML. Desarrollador Web Freelance para EPIMAK (Argentina)
en 2010. 

\Cpp y Zimpl. Desarrollador Freelance para el Gobierno de la 
Ciudad Aut`onoma de Buenos Aires. Desde diciembre 2008 hasta diciembre 2009. 
Proyecto de \href{https://github.com/alexisjawtu/ilp_auction}
{\color{blue}{un software de optimizaci`on}}. 

Analista de datos junior  en Kantar IBOPE
Media (Argentina) desde junio de 2006 hasta diciembre de 2006.

\textbf{Antecedentes Cient`ificos y Publicaciones}\\[6pt]
Miembro del Sakura Science Exchange Program, edici`on 2019. Japan Science and
Technology Agency. Niigata University, Jap`on. Desde el 25 de agosto hasta el 1ro.
de septiembre de 2019.

Investigador Doctoral  -- Full Time con una beca del CONICET,
Argentina. Desde abril 2012 hasta marzo 2017.

\emph{Art`iculos Publicados en Revistas Internacionales con Referato}\\
\href{http://em.rdcu.be/wf/click?upn=lMZy1lernSJ7apc5DgYM8RtiRnX98cgbvE81KQGn5tE-3D_-2Fq09Vpjrycd-2BAOhvYDidHaHWLaG8WMoWs1lRs2mKTzqCwYNFhlGtplH8kb8yUCOrEFESCWAEP1qrD-2BJgg09nu-2Fz61XDXWYdppeXx4JzFRLvI-2FYyjZMrov-2FaxFxLv9MaqfC-2BjYanB-2FkLIArphbTB7hvuq-2BJ-2BP0dpoVrgh2NJYizQcMbyo6AA8jcx6RYsWvb3RMk9c7QXTqmoLaHKr8Xg6yK2lC1IxrYwuPPxXUxfxXQc0WAqTO-2Bg-2F9P-2BWkhJXyogoqkg5GNjl1KqQhJV5xi014g-3D-3D}
{\color{blue}A mixed discretization of elliptic problems on polyhedra using anisotropic hybrid meshes}.\\
Calcolo, CALC-D-18-00054R1 (2019).

\href{https://www.dropbox.com/s/1ai0z0pcyhskpzp/JM-2016.pdf?dl=0}
{\color{blue}An Asymmetric Multi-Item Auction Applied
hasta Internet Service Procurement in Public Schools}.\\
Annals of Operations Investigaci`on - 0254-5330 (2016).

\textbf{Algunas charlas de Investigaci`on en An`alisis Num`erico y Programaci`on
en Congresos Internacionales}\\[6pt]
PyCon Brasil 2019. Ribeir\~ao Preto, Brasil. Octubre 2019.

PyCon Argentina 2019. Buenos Aires, Argentina. Diciembre 2019.

Ninth Workshop on Numerical Analysis of Partial Differential Equations: 
\emph{Santiago Num\'erico III}. Universidad Cat`olica de Chile, Santiago. Junio 2017.

Quinto Congreso de Matem`atica Aplicada, Computacional e Industrial (MACI Argentina).
Sesi`on de Fundamentos de M`etodos Num`ericos y Aplicaciones. Mayo 2015.

Eighth Workshop on Numerical Analysis of Partial Differential Equations:
\emph{La Serena Num\'erica II}. Universidad de La Serena, Chile. Enero 2015.

PyCon Argentina 2018. Buenos Aires, Argentina. Noviembre 2018.

PyDay Rosario 2018. Universidad Nacional de Rosario, Argentina, 27 de octubre
de 2018.

\textbf{Antecedentes de Investigaci`on y Docencia Universitaria}\\[6pt]
Profesor Adjunct Professor -- Part Time. Departamento  de Matem`atica y Sciences,
Universidad de San Andr\'es, Argentina. Desde marzo 2019 hasta agosto 2020.

Jefe de Trabajos Pr`acticos. Full Time Investigaci`on y Teaching Position. 
Departamento  de Matem`atica, Universidad de Buenos Aires. Desde abril 2013 hasta
febrero 2020.

Jefe de Trabajos Pr`acticos -- Part Time. Technology Institute de Buenos Aires (ITBA).\\
Desde agosto 2018 hasta diciembre 2018.

Adjunct Professor -- Part Time. Universidad Abierta Interamericana, Argentina.\\
Desde agosto 2012 hasta marzo 2013.

Adjunct Professor -- Part Time. National Universidad de Avellaneda, Argentina.
Desde febrero 2012 hasta junio 2012.

Junior Teaching Assistant -- Part Time. Departamento  de Matem`atica, Universidad de Buenos Aires. Desde marzo 2010 hasta febrero 2012.
