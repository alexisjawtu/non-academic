\documentclass{res} %[11pt]
\usepackage{palatino}
\usepackage{fancyhdr}
\usepackage{xcolor}
\usepackage[pdftex,colorlinks=true]{hyperref} %\href{url}{text}

\setlength{\textheight}{9.5in}
\newsectionwidth{0pt}
\pagenumbering{arabic}
\pagestyle{plain}
\fancyhf{}

\newcommand{\Cpp}{C\ensuremath{+\!\,\!\,\!\,+\,\,}}

\begin{document}

\name{Dr. Alexis Jawtuschenko, PhD.}
\begin{resume}
\bigskip
 
\textbf{Personal Data}\\[6pt]
National ID: 29006449.\\
Birth date: August, 16th. 1981.\\
Address: Vidal 2027 12th floor E, (1429) City of Buenos Aires. Argentina.\\
Phone number: +54 9 11 5 040 0483.\\
E-mail: alexis.jawtuschenko@gmail.com\\
Web pages: \href{https://github.com/alexisjawtu}{{\color{blue}GitHub profile}}
-- \href{https://www.linkedin.com/in/alexis-jawtuschenko/}{{\color{blue}LinkedIn profile}}

\textbf{University Degrees}\\[6pt]
PhD. at the University of Buenos Aires, Mathematical Sciences area (2018).\\
Thesis topic: Three Dimensional Meshes and Vectorial Numerical Analysis.\\
\href{https://cms.dm.uba.ar/academico/carreras/doctorado/thesisJawtuschenko.pdf}
{{\color{blue}Link to the PhD Thesis.}}

MSc. in Mathematical Sciences at the Univesity of Buenos Aires (2010).\\
Thesis topic: Integer Linear Programming and Operations Research applied to Auctions.\\
\href{https://cms.dm.uba.ar/academico/carreras/licenciatura/tesis/2010/Jawtuschenko_Alexis.pdf}
{{\color{blue}Link to the Graduation Thesis.}}

\textbf{Skills in Natural Languages}\\[6pt]
I am fluent in Spanish, English and German.

\textbf{Skills in Programming Languages and Computing}\\[6pt]
C\masmas,\;\;{Python/NumPy/Pandas}.

{GNU Octave/MATLAB},\;\;{IBM ILOG CPLEX},\;\;{Zimpl}.

{LaTeX}.

{GNU/Linux Debian},\;\;{Git},\;\;{Bash}.

\emph{Specialization Course}
Workshop on Programming Technics for the Science 2017 en FACET, 
University Nacional de Tucum\'an, consistente en teor\'ia y aplicaci\'on de t\'ecnicas 
de desarrollo en C, Python, Fortran y C\masmas\,\,para problemas cient\'ificos.
Completed with a grade of 100/100.

My \href{https://projecteuler.net/profile/alexisj.png}
        {\color{blue}{Project Euler}}
Status.
	

\textbf{Work Background in Programming and Informatics}\\[6pt]
C\masmas 3D Geometry and Meshing Software Developer at Niigata University
(Japan). 
From August 2022 to March 2023, as remote Freelancer.
From June 2023 to August 2023, with a Full Time position on--site as Visiting
Professor/Researcher.
In this role I developed a 3D mesher to model thin silicon wafers in C\masmas
with a visualization module in Python, that was transferred to the industry of
electronic microcircuits in Japan.

Python/Pandas and CPLEX Software Engineer at Mercado Libre (Argentina). 
Full-Time position from August 2020 to June 2022. In this role I developed an 
optimization standalone tool to find solutions to the
Operations Research Staffing Problem in fulfillment centers, with
a Linear Programming model for IBM ILOG CPLEX and handmade Dynamic Programming
algorithms to boost performance. I was responsible of the product and also
trained Junior Analysts.

Php 5 and PostgreSQL Full Time SSr. Software Engineer at the
National Ministry of Education of Argentina (2011).\\[4pt]

ActionScript 3.0 Junior Programming Trainee at DeveGo SA. (Argentina)
Three months training in on--line games development from 
%-Programador ActionScript 3, Python en Devego SA. desde marzo de 2011 hasta mayo de 2011.\\[4 pt]

PHP, MySQL and HTML Part Time Developer at ViaRural.com SA. From June 2008
until February 2011.
Successfully wrote two traditional and basic e-commerce AMD projects
that ended in production.

Freelance PHP and MySQL Web Software Developer at EPIMAK (Argentina)
during three months of 2010. Succesfully wrote an AMD--like small 
platform for users and a shopcart.

Freelance C\masmas and Zimpl developer at the Government of the 
Autonomous City of Buenos Aires, from December 2008 to December 2009. 
In this project I built \href{https://github.com/alexisjawtu/ilp_auction}
{\color{blue}{an optimization software}} to solve a public
\emph{multi--unit} combination auction, by means of implementing
an Integer Linear Programming model to allocate the winners.

Junior Data Analyst -- Data Entry -- Software Mantainer at Ibope S.A.
(Argentina) from June 2006 to December 2006.

\textbf{Scientific Background and Publications}\\[6pt]
Sakura Science Exchange Program, SSP2019 Japan Science and Technology Agency.
Niigata University, Japan. From August, 25th. to September, 1st. 2019.

Doctoral Researcher -- Full Time with a Fellowship of the Ministry of Science from April 2012 until March 2017.

\emph{Articles Published in International Peer--Reviewd Journals}
\href{http://em.rdcu.be/wf/click?upn=lMZy1lernSJ7apc5DgYM8RtiRnX98cgbvE81KQGn5tE-3D_-2Fq09Vpjrycd-2BAOhvYDidHaHWLaG8WMoWs1lRs2mKTzqCwYNFhlGtplH8kb8yUCOrEFESCWAEP1qrD-2BJgg09nu-2Fz61XDXWYdppeXx4JzFRLvI-2FYyjZMrov-2FaxFxLv9MaqfC-2BjYanB-2FkLIArphbTB7hvuq-2BJ-2BP0dpoVrgh2NJYizQcMbyo6AA8jcx6RYsWvb3RMk9c7QXTqmoLaHKr8Xg6yK2lC1IxrYwuPPxXUxfxXQc0WAqTO-2Bg-2F9P-2BWkhJXyogoqkg5GNjl1KqQhJV5xi014g-3D-3D}
{\color{blue}A mixed discretization of elliptic problems on polyhedra using anisotropic hybrid meshes}, CALCOLO, CALC-D-18-00054R1 (2019)

\href{https://www.dropbox.com/s/1ai0z0pcyhskpzp/JM-2016.pdf?dl=0}
{\color{blue}An Asymmetric Multi-Item Auction with Quantity Discounts Applied
to Internet Service Procurement in Buenos Aires Public Schools}, 0254-5330 - 
Annals of Operations Research (2016)

\textbf{Some Talks About Research in Numerical Analysis and Programming given
in International Congresses}\\[6pt]
Decimoprimera conferencia Argentina de Python, PyConAr 2019. Centro Cultural San Mart\'in, diciembre de 2020.

Ninth Workshop on Numerical Analysis of Partial Differential Equations: Santiago Num\'erico III. PUC, Santiago, Chile, June 2017.

V Congress of MACI (Applied, Computational and Industrial Mathematics). Thematic Session of Fundamentals of Numerical Methods and Applications.
Mayo 2015.

Eighth Workshop on Numerical Analysis of Partial Differential Equations: La Serena Num\'erica II. University de La Serena, Chile, January 2015.

PyCon Brazil 2019. Ribeir\~ao Preto, Brazil. October 2019.

PyCon Argentina 2018. Buenos Aires, Argentina. Noviembre 2018.

PyDay Rosario 2018. National University of Rosario, Argentina. October, 27th.
2018.

\textbf{University Teaching and Research Background}\\[6pt]
Adjunct Professor -- Part Time. Department of Mathematics and Sciences 
University of San Andr\'es, Argentina. From March 2019 to August 2020.

Head Teaching Assistant. Full Time Research and Teaching Position. 
Department of Mathematics, University of Buenos Aires. From April 2013 to
February 2020.

Head Teaching Assistant -- Part Time. Department of Mathematics, Technology 
Institute of Buenos Aires. From August 2018 to December 2018.

Adjunct Professor -- Part Time. Universidad Abierta Interamericana, Argentina.
From August 2012 to March 2013.

Adjunct Professor -- Part Time. National University of Avellaneda, Argentina.
From February 2012 to June 2012.

Junior Teaching Assistant -- Part Time. Department of Mathematics, University of Buenos Aires. From March 2010 to February 2012.


\end{resume}
\end{document}

TODO: 26/02/2025 Poner links a lo  que escribi para la UBA. Poisson, guias de
      ejercicios (optim, mefa, etc). El texto de combin. aucti. Los teoremas de An I













------------------------------------------------------------------------------
\textbf{Current Occupation}\\[6pt]
  Freelance programmer
------------------------------------------------------------------------------


------------------------------------------------------------------------------
Trabajos Publicados en Actas de Conferencias\\

A. Jawtuschenko, A. Lombardi: \emph{Estimaciones de interpolaci\'on en pir\'amides con aplicaciones a problemas mixtos con singularidades de arista y v\'ertice}, Res\'umenes de Comunicaciones UMA 2016, p. 1, LXV Reuni\'on de Comunicaciones Cient\'ificas.
\\[4pt]

A. Jawtuschenko, A. Lombardi: \emph{Anisotropic Estimates for $H(\textbf{curl})$ and $H(\textrm{div})$ -- Conforming Elements On Prisms and Applications}, Trabajos presentados al V MACI 2015. MACI Vol. 5 pp. 141 -- 142.\\[4pt]

A. L. Lombardi, A. Jawtuschenko. {\it Estimaciones de interpolaci\'on
para elementos finitos vectoriales sobre ma\-llas poli\'edricas anisotr\'opicas}. Cuaderno de Res\'umenes de Comu\-ni\-caci\-on\-es UMA 2015, LXIV Reuni\'on de Comunicaciones Cient\'ificas.\\[4pt]

A. Jawtuschenko, A. Lombardi: \emph{Estimaci\'on Anisotr\'opica para Elementos de N\'ed\'elec Prism\'aticos}, Cuaderno de Res\'umenes de Comunicaciones UMA 2014, p. 35, LXIII Reuni\'on de Comunicaciones 
Cient\'ificas. 
--------------------------------------------------------------------


--------------------------------------------------------------------
\textbf{Antecedentes De Divulgaci\'on de la Matem\'atica}\\[6pt]
Semana de la Matem\'atica,  Departamento de Matem\'atica, FCEyN, UBA,
ediciones entre 2005 y 2019.\\[4pt]
El Departamento de Matem\'atica en la Feria del Libro 2019.\\[4pt]
Un Festival de Ciencia, Centro Cultural Konex, 2018.\\[4pt]
Exactas en La Noche de los Museos, 2017.\\[4pt] 
Exactas en Tecn\'opolis, 2012.\\[4pt]
IV Festival de Matem\'atica de la Reuni\'on anual de la Uni\'on Ma\-te\-m\'a\-ti\-ca
Argentina en la Academia Nacional de Ciencias, Ciudad de C\'ordoba,
edici\'on 2012.
--------------------------------------------------------------------
