\documentclass{res} %[11pt]
\usepackage{palatino}
\usepackage{fancyhdr}
\usepackage{xcolor}
\usepackage[pdftex,colorlinks=true]{hyperref} %\href{url}{text}

\setlength{\textheight}{9.5in}
\newsectionwidth{0pt}
\pagenumbering{arabic}
\pagestyle{plain}
\fancyhf{}

\newcommand{\Cpp}{C\ensuremath{+\!\,\!\,\!\,+\,\,}}

\begin{document}

\name{Dr. Alexis Jawtuschenko, PhD.}
\begin{resume}
\bigskip
 
\textbf{Personal Data}\\[6pt]
National ID: 29006449.\\
Birthdate: August, 16th. 1981.\\
Address: Vidal 2027 12th floor E, (1428) City of Buenos Aires. Argentina.\\
Phone number: +54 9 11 5 040 0483.\\
E-mail: alexis.jawtuschenko@gmail.com\\
Web pages: \href{https://github.com/alexisjawtu}{{\color{blue}GitHub profile}}
-- \href{https://www.linkedin.com/in/alexis-jawtuschenko/}{{\color{blue}LinkedIn profile}}

\textbf{University Degrees}\\[6pt]
PhD. at the University of Buenos Aires, Mathematical Sciences area (2018).\\
Thesis topic: Three Dimensional Meshes and Vectorial Numerical Analysis.\\
\href{https://cms.dm.uba.ar/academico/carreras/doctorado/thesisJawtuschenko.pdf}
{{\color{blue}Link to the PhD Thesis.}}

MSc. in Mathematical Sciences at the Univesity of Buenos Aires (2010).\\
Thesis topic: Integer Linear Programming and Operations Research applied to Auctions.\\
\href{https://cms.dm.uba.ar/academico/carreras/licenciatura/tesis/2010/Jawtuschenko_Alexis.pdf}
{{\color{blue}Link to the Graduation Thesis.}}

\textbf{Skills in Natural Languages}\\[6pt]
I am fluent in Spanish, English and German.

\textbf{Skills in Programming Languages and Computing}\\[6pt]
\Cpp,\;\;{Python/NumPy/Pandas}.

{GNU Octave/MATLAB},\;\;{IBM ILOG CPLEX},\;\;{Zimpl}.

{LaTeX}.

{GNU/Linux Debian},\;\;{Git},\;\;{Bash}.

\emph{Specialization Course}\\
Workshop on Programming Techniques for the Science at the National University
of Tucum\'an, Argentina (2017). Theory and practice of development topics in 
C, \Cpp, Python and Fortran for scientific problems. Completed with a grade of 100/100.

Link to my \href{https://projecteuler.net/profile/alexisj.png}
        {\color{blue}{Project Euler}}
Status.

\textbf{Work Background in Programming and Informatics}\\[6pt]
\Cpp 3D Geometry and Meshing Software Developer at Niigata University (Japan).\\
From August 2022 to March 2023 as remote Freelancer.\\
From June 2023 to August 2023 with a Full Time position on--site as Visiting
Professor/Researcher.\\
In this role I developed a 3D mesher to model thin silicon wafers in \Cpp
with a visualization module in Python, that was transferred to the industry of
electronic microcircuits in Japan.

Python/Pandas and CPLEX Software Engineer at Mercado Libre (Argentina). 
Full-Time position from August 2020 to June 2022. In this role I developed an 
optimization standalone tool to find solutions to the
Operations Research Staffing Problem in fulfillment centers, with
a Linear Programming model for IBM ILOG CPLEX and handmade Dynamic Programming
algorithms to boost performance. I was responsible of the product and also
trained Junior Analysts.

Php 5 and PostgreSQL Full Time Software Developer at the
National Ministry of Education of Argentina. From June 2011 to
March 2012. 

ActionScript 3.0 and Python Junior Programming Trainee at DeveGo SA. (Argentina).
Three months training for on--line games development from March 2011 to May 2011.

PHP, MySQL and HTML Part Time Developer at ViaRural.com SA. From June 2008
until February 2011.
Successfully wrote two traditional and basic e-commerce AMD projects
that ended in production.

Freelance PHP and MySQL Web Software Developer at EPIMAK (Argentina)
during three months of 2010. Succesfully wrote an AMD--like small 
platform for users and a shopcart.

Freelance \Cpp and Zimpl developer for the Government of the 
Autonomous City of Buenos Aires, from December 2008 to December 2009. 
In this project I built \href{https://github.com/alexisjawtu/ilp_auction}
{\color{blue}{an optimization software}} to solve a public
\emph{multi--unit} combination auction, by means of implementing
an Integer Linear Programming model to allocate the winners.

Junior Data Analyst -- Data Entry -- Software Mantainer at Kantar IBOPE
Media (Argentina) from June 2006 to December 2006.

\textbf{Scientific Background and Publications}\\[6pt]
Member of the Sakura Science Exchange Program, 2019 edition. Japan Science and
Technology Agency. Niigata University, Japan. From August, 25th. to September, 1st. 2019.

Doctoral Researcher -- Full Time with a Fellowship of the Ministry of 
Science of Argentina from April 2012 until March 2017.

\emph{Articles Published in International Peer--Reviewd Journals}\\
\href{http://em.rdcu.be/wf/click?upn=lMZy1lernSJ7apc5DgYM8RtiRnX98cgbvE81KQGn5tE-3D_-2Fq09Vpjrycd-2BAOhvYDidHaHWLaG8WMoWs1lRs2mKTzqCwYNFhlGtplH8kb8yUCOrEFESCWAEP1qrD-2BJgg09nu-2Fz61XDXWYdppeXx4JzFRLvI-2FYyjZMrov-2FaxFxLv9MaqfC-2BjYanB-2FkLIArphbTB7hvuq-2BJ-2BP0dpoVrgh2NJYizQcMbyo6AA8jcx6RYsWvb3RMk9c7QXTqmoLaHKr8Xg6yK2lC1IxrYwuPPxXUxfxXQc0WAqTO-2Bg-2F9P-2BWkhJXyogoqkg5GNjl1KqQhJV5xi014g-3D-3D}
{\color{blue}A mixed discretization of elliptic problems on polyhedra using anisotropic hybrid meshes}.\\
Calcolo, CALC-D-18-00054R1 (2019).

\href{https://www.dropbox.com/s/1ai0z0pcyhskpzp/JM-2016.pdf?dl=0}
{\color{blue}An Asymmetric Multi-Item Auction Applied
to Internet Service Procurement in Public Schools}.\\
Annals of Operations Research - 0254-5330 (2016).

\textbf{Some Talks About Research in Numerical Analysis and Programming given
in International Congresses}\\[6pt]
PyCon Brazil 2019. Ribeir\~ao Preto, Brazil. October 2019.

PyCon Argentina 2019 (as collaborator). Buenos Aires, Argentina. December 2019.

Ninth Workshop on Numerical Analysis of Partial Differential Equations: 
\emph{Santiago Num\'erico III}. Catholic University of Chile, Santiago. June 2017.

Fifth Congress for Applied, Computational and Industrial Mathematics (MACI Argentina).
Thematic Session of Fundamentals of Numerical Methods and Applications. May 2015.

Eighth Workshop on Numerical Analysis of Partial Differential Equations:
\emph{La Serena Num\'erica II}. University of La Serena, Chile. January 2015.

PyCon Argentina 2018. Buenos Aires, Argentina. November 2018.

PyDay Rosario 2018. National University of Rosario, Argentina. October, 27th.
2018.

\textbf{University Teaching and Research Background}\\[6pt]
Adjunct Professor -- Part Time. Department of Mathematics and Sciences,
University of San Andr\'es, Argentina. From March 2019 to August 2020.

Head Teaching Assistant. Full Time Research and Teaching Position. 
Department of Mathematics, University of Buenos Aires. From April 2013 to
February 2020.

Head Teaching Assistant -- Part Time. Technology Institute of Buenos Aires (ITBA).\\
From August 2018 to December 2018.

Adjunct Professor -- Part Time. Universidad Abierta Interamericana, Argentina.\\
From August 2012 to March 2013.

Adjunct Professor -- Part Time. National University of Avellaneda, Argentina.
From February 2012 to June 2012.

Junior Teaching Assistant -- Part Time. Department of Mathematics, University of Buenos Aires. From March 2010 to February 2012.


\end{resume}
\end{document}

Esto lo escribi para Solvd:
1- At MercadoLibre (2020 - 2022) I used a MacBook Pro, because the company had that already decided,
to write a complete optimization tool including a Linear Programming model for some logistics operations
research problems that we had at the depots/warehouses. 
My development framework was Python, IBM CPLEX, Pandas, NumPy, Git, GitHub, Mac Os.
As the stakeholders and final users of the tool were Windows users, for the releases I came with the
idea of getting the company to purchase a single license of Parallels Desktop, and then I was able to fluently
release a simple standalone '.exe' console application, with the help of PyInstaller, to use on 
Windows machines.

2- For the consultation work for my first thesis (2009) that was much simpler because I had a native
Windows operating system installed on my machine (I think it was Windows XP), so I directly compiled 
my project into a console application in an exe file.

Cover letter
I am in love with the development of software for applied mathematics, and particularly in modern C++.
I like the task of enhancing and optimizing the performance of pieces of code, whether they were 
written by me or not, and I can confidently 
state that I am pretty good at it, and that includes the patience and
the perseverance of my code-debugging proficiency. This task is also one that I like, because it has
a strong flavor of the very fundamental activity that a mathematician is supposed to do: to look for
a working solution of an open question, and doing the math starting from the darkness of the ignorance 
about the issue.

TODO: 26/02/2025 Poner links a lo  que escribi para la UBA. Poisson, guias de
      ejercicios (optim, mefa, etc). El texto de combin. aucti. Los teoremas de An I


------------------------------------------------------------------------------
\textbf{Current Occupation}\\[6pt]
  Freelance programmer
------------------------------------------------------------------------------


------------------------------------------------------------------------------
Trabajos Publicados en Actas de Conferencias\\

A. Jawtuschenko, A. Lombardi: \emph{Estimaciones de interpolaci\'on en pir\'amides con aplicaciones a problemas mixtos con singularidades de arista y v\'ertice}, Res\'umenes de Comunicaciones UMA 2016, p. 1, LXV Reuni\'on de Comunicaciones Cient\'ificas.
\\[4pt]

A. Jawtuschenko, A. Lombardi: \emph{Anisotropic Estimates for $H(\textbf{curl})$ and $H(\textrm{div})$ -- Conforming Elements On Prisms and Applications}, Trabajos presentados al V MACI 2015. MACI Vol. 5 pp. 141 -- 142.\\[4pt]

A. L. Lombardi, A. Jawtuschenko. {\it Estimaciones de interpolaci\'on
para elementos finitos vectoriales sobre ma\-llas poli\'edricas anisotr\'opicas}. Cuaderno de Res\'umenes de Comu\-ni\-caci\-on\-es UMA 2015, LXIV Reuni\'on de Comunicaciones Cient\'ificas.\\[4pt]

A. Jawtuschenko, A. Lombardi: \emph{Estimaci\'on Anisotr\'opica para Elementos de N\'ed\'elec Prism\'aticos}, Cuaderno de Res\'umenes de Comunicaciones UMA 2014, p. 35, LXIII Reuni\'on de Comunicaciones 
Cient\'ificas. 
--------------------------------------------------------------------


--------------------------------------------------------------------
\textbf{Antecedentes De Divulgaci\'on de la Matem\'atica}\\[6pt]
Semana de la Matem\'atica,  Departamento de Matem\'atica, FCEyN, UBA,
ediciones entre 2005 y 2019.\\[4pt]
El Departamento de Matem\'atica en la Feria del Libro 2019.\\[4pt]
Un Festival de Ciencia, Centro Cultural Konex, 2018.\\[4pt]
Exactas en La Noche de los Museos, 2017.\\[4pt] 
Exactas en Tecn\'opolis, 2012.\\[4pt]
IV Festival de Matem\'atica de la Reuni\'on anual de la Uni\'on Ma\-te\-m\'a\-ti\-ca
Argentina en la Academia Nacional de Ciencias, Ciudad de C\'ordoba,
edici\'on 2012.
--------------------------------------------------------------------
