\textbf{Personal Data}\\[6pt]
National ID: 29006449.\\
Birth date: August, 16th. 1981.\\
Address: Vidal 2027 12th floor E, (1429) City of Buenos Aires. Argentina.\\
Phone number: +54 9 11 5 040 0483.\\
E-mail: alexis.jawtuschenko@gmail.com\\
Web pages: \href{https://github.com/alexisjawtu}{{\color{blue}GitHub profile}}
-- \href{https://www.linkedin.com/in/alexis-jawtuschenko/}{{\color{blue}LinkedIn profile}}

\textbf{University Degrees}\\[6pt]
PhD. at the University of Buenos Aires, Mathematical Sciences area (2018).\\
Thesis topic: Three Dimensional Meshes and Vectorial Numerical Analysis.\\
\href{https://cms.dm.uba.ar/academico/carreras/doctorado/thesisJawtuschenko.pdf}
{{\color{blue}Link to the PhD Thesis.}}

MSc. in Mathematical Sciences at the Univesity of Buenos Aires (2010).\\
Thesis topic: Integer Linear Programming and Operations Research applied to Auctions.\\
\href{https://cms.dm.uba.ar/academico/carreras/licenciatura/tesis/2010/Jawtuschenko_Alexis.pdf}
{{\color{blue}Link to the Graduation Thesis.}}

\textbf{Skills in Natural Languages}\\[6pt]
I am fluent in Spanish, English and German.

\textbf{Skills in Programming Languages and Computing}\\[6pt]
C\masmas,\;\;{Python/NumPy/Pandas}.\\[4pt]
{GNU Octave/MATLAB},\;\;{IBM ILOG CPLEX},\;\;{Zimpl}.\\[4pt]
{LaTeX}.\\[4pt] 
{GNU/Linux Debian},\;\;{Git},\;\;{Bash}.\\[4pt]
\emph{Specialization Course}
Workshop on Programming Technics for the Science 2017 en FACET, 
Universidad Nacional de Tucum\'an, consistente en teor\'ia y aplicaci\'on de t\'ecnicas 
de desarrollo en C, Python, Fortran y C\masmas\,\,para problemas cient\'ificos.
Completed with a grade of 100/100.

https://projecteuler.net/profile/alexisj.png

\textbf{Work Background in Programming and Informatics}\\[6pt]
Freelancer

NiiUniv presencial

Freelancer NiiUniv

SSr Python/Pandas and CPLEX Software Engineer at Mercado Libre, Buenos Aires, Argentina 
Full-time position from August 2020 to June 2022. 

Php 5 and PostgreSQL SSr. Software Engineer at the
National Ministry of Education of Argentina (2011).\\[4pt]

Freelance C\masmas and Zimpl developer. Project built to solve a public
auction for the Government of the City of Buenos Aires, put as part of my
MSc. graduation thesis.\\[4pt]

Pasante de investigaci\'on en ViaRural.com SA. Programador
Php 4 y MySQL desde junio de 2008 
hasta febrero de 2011. Pasant\'ia de la Facultad de Ciencias Exactas y Naturales, UBA.\\[4pt]
Analista de Datos en Ibope S.A. desde junio de 2006 hasta diciembre de 2006.

\textbf{Scientific Background and Publications}\\[6pt]
Intercambio como investigador con la Universidad de Niigata, Jap\'on,
Sakura Science Exchange Program, SSP2019, patrocinado por
la Japan Science and Technology Agency (JST). Desde el 25 de agosto
hasta el primero de septiembre de 2019.\\[6pt]
Becario de Doctorado del CONICET entre el 2 de abril de 2012 y el 31 de marzo de 2017.\\[6pt]
Art\'iculos Publicados en Revistas Internacionales con Referato:\\[6pt]
\emph{A mixed discretization of elliptic problems on polyhedra using anisotropic hybrid meshes},
CALCOLO, CALC-D-18-00054R1 (2019)\\
\href{http://em.rdcu.be/wf/click?upn=lMZy1lernSJ7apc5DgYM8RtiRnX98cgbvE81KQGn5tE-3D_-2Fq09Vpjrycd-2BAOhvYDidHaHWLaG8WMoWs1lRs2mKTzqCwYNFhlGtplH8kb8yUCOrEFESCWAEP1qrD-2BJgg09nu-2Fz61XDXWYdppeXx4JzFRLvI-2FYyjZMrov-2FaxFxLv9MaqfC-2BjYanB-2FkLIArphbTB7hvuq-2BJ-2BP0dpoVrgh2NJYizQcMbyo6AA8jcx6RYsWvb3RMk9c7QXTqmoLaHKr8Xg6yK2lC1IxrYwuPPxXUxfxXQc0WAqTO-2Bg-2F9P-2BWkhJXyogoqkg5GNjl1KqQhJV5xi014g-3D-3D}{\color{blue}https://rdcu.be/bsfEg}
\\[4pt]
\emph{An Asymmetric Multi-Item Auction with Quantity Discounts Applied to Internet 
Service Procurement in Buenos Aires Public Schools}, 
0254-5330 - Annals of Operations Research (2016)\\
\href{https://www.dropbox.com/s/1ai0z0pcyhskpzp/JM-2016.pdf?dl=0}{\color{blue}https://www.dropbox.com/s/1ai0z0pcyhskpzp/JM-2016.pdf?dl=0}\\[6pt]

\textbf{Some talks about research in Numerical Analysis given in congresses}\\[6pt]
La serena numerica
Decimoprimera conferencia Argentina de Python, PyConAr 2019. Centro Cultural San Mart\'in,
diciembre de 2020 (\emph{colaborador}).\\[4pt]
Conferencia Brasile\~na de Python, PyCon Brasil 2019. Centro de Convenciones de
Ribeir\~ao Preto, octubre de 2019 (\emph{orador}).\\[4pt]
D\'ecima conferencia Argentina de Python, PyConAr 2018.
Centro Cultural San Mart\'in, noviembre de 2018 (\emph{orador}).
PyDay Rosario 2018. Universidad Nacional de Rosario, FCEIA, 27 de octubre
de 2018 (\emph{orador}).

\textbf{University Teaching and Writing Background}\\[6pt]
Profesor de C\'atedra.\\
Departamento de Matem\'atica y Ciencias, 
Universidad de San Andr\'es, desde marzo de 2019.\\[4pt]
Jefe de Trabajos Pr\'acticos.\\Universidad de Buenos Aires,
Ciclo B\'asico Com\'un, desde agosto de 2016.\\[4pt]
Jefe de Trabajos Pr\'acticos, Dedicaci\'on Exclusiva. FCEN, UBA, 
Departamento de Matem\'atica.\\Desde el 19/04/2017 hasta el 29/02/2020.\\[4pt]
%Jefe de Trabajos Pr\'acticos. CBC, UBA. Desde el 12/8/2019.\\[4pt]
Jefe de Trabajos Pr\'acticos. Departamento de Matem\'atica, 
Instituto Tecnol\'ogico de Buenos Aires.\\Desde 08/2018 hasta 12/2018.\\[4pt]
%Ayudante de Primera, Dedicaci\'on Exclusiva. FCEN, UBA, Departamento de 
%Matem\'atica. Per\'iodo: 19/4/2017 -- 22/5/2017.\\[6pt]
Jefe de Trabajos Pr\'acticos, Dedicaci\'on Parcial. Departamento de Matem\'atica, FCEN, UBA.\\
Per\'iodo: 8/2015 a 3/2017.\\[4pt]
Ayudante de Primera, Dedicaci\'on Parcial. CBC, UBA. Per\'iodo: 4/2014 a 9/2014.\\[4pt]
Ayudante de Primera, Dedicaci\'on Parcial. Departamento de Matem\'atica, FCEN, UBA.\\
Per\'iodo: 4/2013 a 7/2015.\\[4pt]
Profesor Adjunto. Universidad Abierta Interamericana.
Per\'iodo: 8/2012 a 3/2013.\\[4pt]
Profesor Adjunto. Universidad Nacional de Avellaneda.
Per\'iodo: 2/2012 a 6/2012.\\[4pt]
Ayudante de Segunda. Departamento de Matem\'atica, FCEN, UBA. Per\'iodo: 3/2010 a 2/2012.\\[6pt]
Redacci\'on de Monograf\'ias:\\[6pt]
